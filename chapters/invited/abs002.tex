\nazev{Linear programming: Old and new results, old and new challenges}

\autor{Michal Černý}{jmeno.prijmeni@nekdevesvete.org}

\odkudjsem{Department of Econometrics, Prague University of Economics and Business, Czech Republic}

\begin{Abstrakt}
Linear programming (with continuous variables) has had a rich history since Dantzig’s formalization of the Simplex Method in 1940’s. From the computational viewpoint, the most significant milestones involve Klee-Minty’s construction of their “cube” from 1973, the surprising 1979’s polynomiality result by Khachiyan on Shor-Nemirovski-Yudin’s Ellipsoid Method, and Karmarkar’s first Interior Point Method from 1984 followed by the intensive development of IPMs in the “IPM decade”. In parallel, the theory of average-case analysis of LP algorithms starts its development, with the pioneering work by Borgwardt, Haimovich and others.

The work still goes on and new results are being reported in literature. We mention at least two results of particular significance: Spielman-Teng’s smoothed analysis of the shadow-vertex Simplex Method and Disser-Skutella’s result on the NP-mightiness of the Simplex Method.

However, many problems still remain open and seem to be difficult. Can the combinatorial diameter of every polytope be bounded by a polynomial? (This is the polynomial version of Hirsch’s conjecture. Recall that the linear \uv{$n-d$} version was disproved by Santos in 2010, but the polynomial version remains open.) If so, can we use this fact constructively to design a pivoting strategy for the Simplex Method to make it a polynomial-time algorithm? Another group of problems arises from the weak polynomiality of the Ellipsoid Method and IPMs. Can we get rid of the unpleasant “Big-L” from their iteration bounds? Does there exist a strongly polynomial algorithm?

There are further research challenges in LP; a (subjective) choice of them is presented in the talk.  One particularly interesting is the following: Can the linear programming problems solvable by the Ellipsoid Method with separation and membership oracle (by Grötschel, Lovász and Schrijver) be converted into polynomial-time IPMs?

\end{Abstrakt}

\klicovaslova{klicove slovo 1, klicove slovo 2, sklicove slovo 3}

\clearpage