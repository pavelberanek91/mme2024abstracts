\hlavka{ Using of the simulated annealing in the decision support system of the railway transport nodes }{ Andrea Galadíková }{ andrea.galadikova@fri.uniza.sk }{ Faculty of Management Science and Informatics, University of Žilina, Slovakia\newline Faculty of Management Science and Informatics, University of Žilina, Slovakia\newline Faculty of Management Science and Informatics, University of Žilina, Slovakia }{ Marek Kvet, Maroš Janovec }

\begin{Abstrakt}
    Rail transportation is an important part of ensuring the transport of people and goods. Among many other requirements, it is important to take care of the maintenance of each train set to ensure smooth operation. This maintenance is carried out in the maintenance depot. Due to the large number of different maintenance activities that need to be carried out on the train sets, but especially due to the limitations associated with the nature of train traffic (mainly movement on rails), it is important to deal with the optimization of individual processes. This is particularly challenging at the operational level, where many unexpected influences enter the planned schedules. Based on previous research dealing with the possibilities of the supporting operational management, in this article we deal with the application of the heuristic method of the simulated annealing to the selected task of assigning the sequence of the individual maintenance activities. This approach was used as part of a decision support system and was validated by a simulation model of a selected maintenance depot.
\end{Abstrakt}

\klicovaslova{
    simulated annealing, railway, decision support
}


\clearpage