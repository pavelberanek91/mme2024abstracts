\hlavka{ Analysis of the impact of the reduction of state support for research and development }{ Eva Vychodilová }{ eva.vychodilova@tul.cz }{ Technical University of Liberec, Czechia\newline Technical University of Liberec, Czechia }{ Michaela Matoušková }

\begin{Abstrakt}
    Although state support for research and development (R\&D) has been growing steadily in recent years, the rate of growth is slowing considerably. Forecasts predict a further year-on-year decline in state R\&D spending, raising concerns about its impact on research projects and innovation activities within higher education, research institutions, and the private sector. This raises a fundamental question: how will universities and firms adapt to this changing environment of public support for R\&D?\newline Extensive research across countries highlights the importance of promoting collaboration and networking between universities (knowledge production sector) and the private sector (application sector). This collaboration between universities and industry is widely regarded as a key driver of innovation, leading to technological progress, product development, and economic growth.\newline This study aimed to model the evolution of time series describing the likely response of universities and firms to changes in government support for R\&D and to identify and model the structural breaks that occur in these series. Having captured the trend of the time series, a prediction of the time series was then made. Chow's forecasting test was chosen as the extrapolation criterion.\newline Declining state support for R\&D requires a deeper understanding of how universities and firms will adapt. Using time series analysis and structural break detection, this study aims to shed light on this crucial question and inform future R\&D policy and university-industry collaboration strategies.
\end{Abstrakt}

\klicovaslova{
    State support, Research and development, Funding, Time series, Model the structural breaks
}


\clearpage