\hlavka{ Economic Growth and Agricultural Sector Dynamics in the Visegrad Group: A Panel Analysis }{ Radmila Krkošková }{ stoklasova@opf.slu.cz }{ Silesian University in Opava, School of Business Administration in Karviná, Department of Informatics and Mathematics, Czechia\newline Silesian University in Opava, School of Business Administration in Karviná, Department of Informatics and Mathematics, Czechia\newline Silesian University in Opava, School of Business Administration in Karviná, Department of Informatics and Mathematics, Czechia }{ Zuzana Neničková, Lucie Waleczek Zotyková }

\begin{Abstrakt}
    This empirical investigation analyzes the interplay between employment in the agricultural sector, food production index, and economic growth within the Visegrad Group. Utilizing panel analysis, the study investigates annual time series data spanning from 2005 to 2023. Employing panel data analysis and the Autoregressive Distributed Lag (ARDL) model, the study aims to clarify the research objectives. Findings reveal that, in the short and long terms, the food production index significantly impacts economic growth. Specifically, an increase in the food production index correlates with boosted economic growth. Conversely, a decline in the agricultural sector labor force tends to spur economic growth.\newline The agricultural sector continues to be a vital pillar of economic growth in the Visegrad Group. Understanding the dynamics, challenges, and opportunities within this sector is essential for policymakers, stakeholders, and investors to formulate strategies that promote sustainable development across the countries.
\end{Abstrakt}

\klicovaslova{
    agricultural, ARDL, economic growth, food production index, panel analyse
}


\clearpage