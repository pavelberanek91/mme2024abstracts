\hlavka{ On the Dependencies between Sentiment and Assets‘ Characteristics }{ Aleš Kresta }{ ales.kresta@vsb.cz }{ VSB - Technical University of Ostrava, Czechia }{  }

\begin{Abstrakt}
    In this empirical study, we investigate the relationships between market sentiment and the characteristics of stock returns. Our analysis focuses on the impact of the sentiment index on returns, volatility, and trading volume. The dataset under analysis comprises the components of the Standard \& Poor’s 500, with adjusted close prices considered. As explanatory variables, we incorporate historical data on Fama-French factors, and as a sentiment proxy, we use the historical data of the Investors Intelligence Survey conducted by the American Association of Individual Investors and the University of Michigan Consumer Sentiment Index. GARCH models are used to measure volatility. The dependencies are assessed through linear regressions with different response variables (return, volatility, trading volume) and explanatory variables (Fama-French factors and sentiment indexes). When applicable, we also utilize the GUHA method of automatic generation of hypotheses based on empirical data. The results indicate a limited influence of sentiment on stock returns, but statistical significance in explaining trading volume. Positive sentiment leads to a decrease in trading volume, and negative sentiment leads to an increase in trading volume. These findings underscore the importance of considering investor sentiment in understanding market dynamics.
\end{Abstrakt}

\klicovaslova{
    sentiment, stock returns, volatility, trading volume
}


\clearpage