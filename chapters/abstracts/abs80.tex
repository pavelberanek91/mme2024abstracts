\hlavka{ Knapsack problem for build spare parts stock }{ Robert Szczyrbak }{ rszczyrbak@agh.edu.pl }{ AGH University of Krakow, Poland }{  }

\begin{Abstrakt}
    The manufacture of products and services involves the operation of machinery and equipment. In the course of operations, machinery gradually wears out, causing product defects, breakdowns or even to accidents. Product defects and failures strongly affect the OEE rate. Machine failures reduce the availability rate. The role of the maintenance department is to ensure the safe and continuous operation of the machinery fleet. For this purpose, the costs of technical materials and labor costs are incurred.\newline Ensuring continuous operation of the machinery park is both the implementation of periodic inspections associated with the replacement of worn-out components, but also the minimization of repair time. The main factor causing an increase in repair time is the lack of possession of the required parts in the technical warehouse.\newline In addition to standard indicators related to logistical aspects, it is worth using mathematical models for this purpose, such as integer programming models. To develop such a model, the research problem should be defined. It can be related both to reducing the value of the warehouse or total reduction of fixed costs, but also to raising the value of the machinery availability index. This paper shows knapsack problem, which can be used in technical warehouse. The main goal of this model is to minimize loses in production due failures.\newline Securing in spare parts of the machinery park is a very important part of the management strategy. Decisions related to whether or not to have a particular component are worth justifying with appropriate arguments formed on the basis of analyses of availability and uniqueness of spare parts. It is also worth remembering that parts in stock are subject to obsolescence and in a few years a given component may not fully perform its functions.
\end{Abstrakt}

\klicovaslova{
    maintenance, Operational research, spare parts, knapsack problem
}


\clearpage