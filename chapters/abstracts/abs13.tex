\hlavka{ Comparative Analysis of Bankruptcy Prediction Models in Metallurgical Industry: Logistic Regression versus Artificial Intelligence Techniques }{ Stanislav Letkovský }{ stanislav.letkovsky@smail.unipo.sk }{ University of Prešov, Slovakia\newline University of Prešov, Slovakia\newline University of Prešov, Slovakia\newline University of Prešov, Slovakia }{ Sylvia Jenčová, Marta Miškufová, Petra Vašaničová }

\begin{Abstrakt}
    Bankruptcy prediction becomes part of the financial manager's toolkit, enabling them to address the potential threat of bankruptcy with the aid of a suitable tool. AI is increasingly becoming a popular method in this area as well. Will this technology supplant classic logistic regression in terms of performance? This study aims to compare the prediction accuracy of LR and selected AI methods. The research is conducted on a sample of over 4,600 enterprises from the metallurgical industry in the conditions of the Slovak Republic from 2019 to 2021. This period allows for a comparison between the pre-crisis period and the period of crisis during the COVID-19 pandemic. Currently, no model focuses on this specific industry in the conditions of this country. This study offers a unique tool for identifying bankruptcy in the metallurgical industry of the Slovak Republic, which can be easily adapted to other countries with a similar underdeveloped capital market. A critical aspect of bankruptcy prediction is the selection of reliable predictors. Based on the analyzed literature, 40 financial indicators are empirically investigated. The proposed prediction models contain optimally selected indicators potentially significant in predicting bankruptcy under these conditions. All proposed models achieve high accuracy.
\end{Abstrakt}

\klicovaslova{
    bankruptcy prediction, artificial intelligence, logistic regression, Slovak metallurgical industry, financial indicators
}


\clearpage