\hlavka{ Construction of a new DEA-based Compo-site Index for Circular Economy Assessment in the EU }{ Markéta Matulová }{ 8987@mail.muni.cz }{ ESF, Masaryk University, Brno, Czechia }{  }

\begin{Abstrakt}
    In recent years, there has been growing interest in exploring the concept of the circular economy as a potential solution for enhancing the sustainability of our economic system. The development of circular economy indicators provides valuable insights allowing the evaluation of the progress on the path to circularity and sustainability. On the other hand, composite indicators often stir controversies about the unavoidable subjectivity that is connected with their construction. Usually, the normalized sub-indicators are just added, sometimes with certain weights associated with the sub-indicators. We will depart from that approach and compute alternate composite index for 28 EU countries using flexible weights obtained by Data Envelopment Analysis. Using flexible weighting can promote buy-in from relevant stakeholders, making the final results more widely accepted. Additionally, DEA-based indicator provides more information on the relative performance of evaluated units and offers implications such as identifying target values of sub-indicators or selecting peer units for benchmarking purposes.
\end{Abstrakt}

\klicovaslova{
    Circularity, Composite Index, Data Envelopment Analysis
}


\clearpage