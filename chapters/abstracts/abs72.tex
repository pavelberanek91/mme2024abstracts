\hlavka{ THE LONG-TERM-CARE CAREGIVER ROUTING AND SCHEDULING PROBLEM WITH CONSIDERATIONS OF STOCHASTIC TRAVEL TIMES AND SERVICE TIMES }{ Jenn-Rong Lin }{ jrlin@email.ntou.edu.tw }{ National Taiwan Ocean University, Taiwan }{  }

\begin{Abstrakt}
    With increasing elderly citizens yearly, there are more and more elderly people need long-term care (LTC) service. The long-term care service institutions are responsible for providing home-based LTC service according to the plan setting up by the responsible integrated community service center and family members of the elderly in need. This is a huge challenge to long-term care service institutions since there is a growing demand of home-based LTC service but with insufficient home-based LTC caregivers for providing service.  Efficient and effective caregiver routes and schedules are essential to conquer the challenge. The robust planning of home-based LTC caregiver routes and schedules is a very intricate engineering task, since it requires considerations of the management of caregivers’ working rules and workloads, the matching of caregivers’ skills and customers’ service requests and their preferences and complicated analysis among the selection of customers, the staffing of full-time and part-time caregivers, and numerous time-window and space constraints which are highly correlated with each other in a plan horizon. Therefore, we aim to develop a robust planning model to help LTC service institutions plan home-based caregiver routes and schedules. We first formulate and analyze a deterministic planning model for home-based LTC caregiver routing and scheduling with the considerations of working rules, workloads, service items, service time windows and service route duration. We then formulate and analyze a robust planning model for home-based LTC caregiver routing and scheduling in urban areas with the additional considerations of stochastic travel times and stochastic service times and penalty costs for violating working rules. A scenario-based stochastic model is proposed for this problem. A meta-heuristics embedded with a routing heuristic is developed to solve the model. Numerical examples are created to illustrate and to test the proposed models.
\end{Abstrakt}

\klicovaslova{
    Long term care, Caregiver, routing and scheduling, Stochastic vehicle routing
}


\clearpage