\hlavka{ A Comparative Analysis of S-VAR and Traditional Filtering Methods in Output Gap Estimation }{ Dominik Kavrik }{ kavd00@vse.cz }{ Prague University of Economics and Business, Czechia }{  }

\begin{Abstrakt}
    The Taylor curve, illustrating the trade-off between inflation variability and output gap fluctuations, is pivotal in shaping monetary policy decisions. Traditional analyses often utilize the Hodrick-Prescott filter to estimate the output gap, which may introduce distortions affecting policy interpretation. This paper proposes an alternative approach by employing the Structural Vector Autoregression (S-VAR) model to filter the output gap and compares its effectiveness against standard methods including the Hodrick-Prescott, Christiano-Fitzgerald, and Beveridge-Nelson techniques. This comparative analysis aims to uncover how different filtration methods influence the stability and accuracy of the Taylor curve estimates. The results suggest that the choice of filtering technique not only significantly alters the perceived efficacy of monetary policy but also necessitates a reassessment of methodological preferences in macroeconomic analysis. This study underscores the importance of selecting robust filtering tools in the empirical evaluation of key macroeconomic relationships.
\end{Abstrakt}

\klicovaslova{
    Taylor Curve, Monetary Policy, Output Gap Estimation
}


\clearpage