\hlavka{ An Empirical Efficiency Comparison of Downside Risk and Drawdown Risk in Dynamic Portfolio Optimization }{ Qian Gao }{ qian.gao.st@vsb.cz }{ VSB-Technical University of Ostrava, Czechia }{  }

\begin{Abstrakt}
    Through an in-depth exploration of risk, we have come to recognize the crucial role of downside risk in identifying potential extreme risks, and drawdown risk focuses on losses under adverse conditions. These two types of risks contribute to decision optimization, but their actual benefits within portfolios remain unclear. Consequently, this paper aims to utilize empirical data from diverse market environments, employing these two risk types to construct dynamic portfolios and conduct multi-dimensional comparisons. The chosen risk measures comprise maximum drawdown, conditional value at risk, entropic value at risk, conditional drawdown at risk, and entropic drawdown at risk. In our empirical study, we demonstrate the performance of downside risk and drawdown risk under different market conditions, comparing indicators of optimal holding periods for maximum returns across various time periods using Z-scores. The results indicate the effectiveness of risk-constructed portfolios, with drawdown risk exhibiting notable advantages. Additionally, the efficiency varies across different time periods and market conditions. This further elucidates that various markets may possess unique risk measurements tailored to their respective characteristics.
\end{Abstrakt}

\klicovaslova{
    portfolio optimization, downside risk, drawdown risk, time periods
}


\clearpage