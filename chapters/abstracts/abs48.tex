\hlavka{ Comparison of Python Metaheuristic Packages }{ Vojtěch Vávra }{ vojtech.vavra@vse.cz }{ Prague University of Economics and Business - Faculty of Informatics and Statistics, Czechia }{  }

\begin{Abstrakt}
    Metaheuristic approaches are utilized to find sub-optimal solutions within a reasonable timeframe. This is crucial for NP-hard problems such as the traveling salesman problem, the vehicle routing problem and the knapsack problem. However, identifying the appropriate software to execute such algorithms is not straightforward. This paper presents a comprehensive study aimed at identifying a suitable package in the programming language Python. Python is one of the most widely used programming languages worldwide and is employed daily by numerous companies. The sheer number of packages available can be overwhelming, making it challenging to select the right tool for a given problem.The objective of this paper is to locate and compare such packages, preselect suitable ones and determine the best package or packages. The criteria for decisionmaking include: first, the number of algorithms implemented; second, flexibility, customization and tuning capabilities; third, performance; fourth, quality of documentation; fifth, the learning curve in relation to the knowledge of AI; sixth, maintenance; seventh, community support and finally, the overall health of the package.
\end{Abstrakt}

\klicovaslova{
    Python, metaheuristics, DEAP, MEALPY, NiaPy, Opytimizer, PyGmo
}


\clearpage