\hlavka{ Level of efficiency of the energy industry in EU countries }{ Ondřej Novák }{ xnovak.ondrej@gmail.com }{ Mendel University in Brno, Czechia\newline Mendel University in Brno, Czechia }{ Veronika Blašková }

\begin{Abstrakt}
    This article compares the efficiency of companies in the energy sector in EU coun-tries. The efficiency calculation is performed through data envelopment analysis method. The dataset includes accounting data (annual frequency) for a total of 3893 companies. The output variables represent the turnover of the company and also the net income of the company. The inputs include variables representing the capital factor and the labor factor.\newline Empirical results show that the efficiency of this sector in EU countries is at a rela-tively high level (the average is around 75%). Countries such as Italy and Bulgaria have a significant share of efficient companies. Denmark, on the other hand, has the least. According to the average (and median values), companies in Estonia and Denmark performed very well. Both of these countries have a high share of renew-able energy, which may have had a positive impact on their performance.
\end{Abstrakt}

\klicovaslova{
    accounting data, company, data envelopment analysis, efficiency
}


\clearpage