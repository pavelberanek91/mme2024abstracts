\hlavka{ Applying the FIML dynamic structural model in tourism industry }{ Lukáš Malec }{ lukas.malec@vse.cz }{ Prague University of Economics and Business, Prague, Czech Republic, Czechia }{  }

\begin{Abstrakt}
    Even though the standard approaches of structural techniques can offer pseudo-maximum likelihood estimates in time series data, better criteria are needed under current technological development. The dynamic structural equation model (DSEM) used in this study originates from Ciraki, D. (2007): Dynamic structural equation model, Estimation and inference. This procedure enables the lagged latent endogenous as well as exogenous variables to arrive at a solution in one process, together with variances of model errors. We constructed one variance-covariance matrix for the entire vectorized dataset, and the likelihood is then evaluated for a single observation. Because such matrices generally suffer from numerical problems, regularization has been introduced. The initial evaluation using the 3SLS approach based on stationary methodology and identification are both performed in the observed form. Despite the indisputable advantages of the method, computational difficulty is probably the reason for the full dynamic system not yet being incorporated into econometric packages.
\end{Abstrakt}

\klicovaslova{
    Dynamic latent systems, structural equations, tourism
}


\clearpage