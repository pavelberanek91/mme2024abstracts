\hlavka{ Comparing Measures of Product Relatedness on Data from Czech Drugstore Retail Chain }{ Petr Krautwurm }{ petr.krautwurm@vse.cz }{ Prague University of Economics and Business, Czechia }{  }

\begin{Abstrakt}
    This article examines Robinson's elasticity of substitution estimator alongside cross-price elasticity, from both theoretical and empirical perspectives, focusing on their application in classifying products as substitutes or complements. Initially, we illustrate the theoretical interconnections between these measures, demonstrating that they should consistently classify products similarly. Furthermore, we introduce a novel approach to adjust for the presence of perfect substitutes in individual transactional datasets, a common challenge in economic analyses. By aggregating perfect substitutes, we maintain the integrity of product relations, thus enhancing the utility of the data without introducing biases. Utilizing this adjusted dataset, we apply both measures to assess product relatedness and find that they effectively and consistently classify products as either substitutes or complements, validating theoretical predictions and our methodological innovations.
\end{Abstrakt}

\klicovaslova{
    Elasticity of substitution, Cross-price elasticity, Individual data, Substitutes, Complements
}


\clearpage