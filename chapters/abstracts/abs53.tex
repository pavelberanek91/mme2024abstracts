\hlavka{ On-line learning process for setting of heuristic parameters }{ Jaroslav Janacek }{ jaroslav.janacek@fri.uniza.sk }{ Zilinska univerzita v Ziline, Slovakia\newline University of Žilina - Faculty of Management Science and Informatics, Slovakia }{ Marek Kvet }

\begin{Abstrakt}
    Tuning of sophisticated optimization heuristics represents a substantial part of the heuristic application and it decides on final success or fail of the application. Tuning of a heuristic is usually based on proper setting of heuristic parameters at such values, which ensure the most efficient run of the heuristic. The admissible values of the parameters are known in advance only in rear cases. Mostly, they must be determined for each individual case separately. It can be performed by previous research during the phase of heuristic tuning or by a self-learning process, which is a part of regular heuristic performance. Within this paper, an on-line learning process applied to swap heuristic parameter setting is studied. The swap heuristic is run in the frame of the gradual refinement process assigned to the problem of Pareto front approximation. The heuristic environment assures frequent repeating of the heuristic run and thus, the learning process may lead to significant results. The issue of the learning process may be either a recommendable parameter value or it can be found that the parameter belongs to the class of sensitive parameters and no recommendable value exists.
\end{Abstrakt}

\klicovaslova{
    Location problems, Pareto front approximation, heuristics, online learning process
}


\clearpage