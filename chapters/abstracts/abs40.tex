\hlavka{ Simulation and strategy for the Secretary problem with cardinal function based on job specifications }{ Petr Chládek }{ chladek@ef.jcu.cz }{ University of South Bohemia in České Budějovice/Faculty of Economics, Department of Applied Mathematics and Informatics, Czechia\newline University of South Bohemia in České Budějovice/Faculty of Education, Department of Mathematics, Czechia\newline Centre of Information Technologies, Branišovská 31a, CZ - 371 15, České Budějovice, Czech Republic, Czechia\newline University of West Bohemia/Faculty of Economics, Department of Finance and Accounting, Czechia\newline University of South Bohemia in České Budějovice/Faculty of Education, Department of Mathematics, Czechia }{ Marika Hrubešová, Štěpán Mudra, Martin Polívka, Tomáš Roskovec }

\begin{Abstrakt}
    The classical Secretary problem is an applied mathematical model for choosing the best applicant from a sample of a given size; the decider can not return to an already rejected candidate and does not a priori know the scale of quality of the candidates. In this well-known problem with a given number of data with unknown distributions, we should stop the search once we consider a number to be the highest one. We opt for the original motivation, but rather than attempting to choose the best one and fail in most cases, we suggest a strategy to control the mean value of the chosen one. The utility of the candidates in different experiments is designed based on the theoretic distribution for specific job titles; we use a cardinal function with the argument being the candidate's percentile among the population. Based on the experiment, we present strategies for hiring for different roles and reveal that the strategy has to be specific for a job; otherwise, it could be suboptimal. Our strategies are based on two parameters: the size of the sample part we examine and reject at the beginning of the search and the percentile from the examined sample we demand the candidate to outperform to be accepted. We simulate all possibilities of ordering 12 candidates and evaluate the strategies based on the experiment.
\end{Abstrakt}

\klicovaslova{
    Secretary problem, Hiring strategy, Cardinal utility function, Decision-making in management
}


\clearpage