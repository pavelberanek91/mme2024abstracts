\hlavka{ Refining Fourier Approach with Constrained Parameter Estimation and Penalizing Seasonal Distortions }{ Lukáš Veverka }{ lukyveverka@seznam.cz }{ University of Economics in Prague, Czechia }{  }

\begin{Abstrakt}
    This study addresses the challenge of determining initial parameters in a constrained space for Fourier transformations applied to decompose time series data of media investments, focusing on identifying the underlying seasonal components at minimal levels. Recognizing that observed peaks of any business KPIs can be attributed to various external factors such as special events and media activities. The research modifies the Sum of Squared Residuals (SSR) methodology, which aims to penalize overestimations in the form of negative residuals. Thus, the distribution of residuals becomes highly skewed. The maximum likelihood method is used to get likelihood, and the Akaike information criterion (AIC) is used to evaluate the appropriate order of Fourier transformation. Based on the outputs, it is possible to provide suitable initial parameters for more complex regression models based on non-linear optimization.
\end{Abstrakt}

\klicovaslova{
    Fourier transformation, Constrained parameter estimation, Seasonal decomposition, Data-driven marketing
}


\clearpage