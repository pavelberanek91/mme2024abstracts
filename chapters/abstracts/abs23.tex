\hlavka{ Level of efficiency of the tertiary education sector in the EU }{ Ondřej Novák }{ xnovak.ondrej@gmail.com }{ Mendel University in Brno, Czechia\newline Mendel University in Brno, Czechia }{ Michaela Staňková }

\begin{Abstrakt}
    Despite the common legislative framework of the European Union, there are differ-ences in the education system among its members. Given that education is funded from public budgets, it should be monitored whether these funds are used efficient-ly. Using a non-parametric method of data envelopment analysis, we evaluate the tertiary education systems of EU countries based on the number of graduates with respect to the inputs used. In addition to the staff, we have also included public re-sources (namely R\&D expenditure and public expenditure on tertiary education) as inputs. Empirical results show that some developed countries (such as Germany and Austria) do not have an efficient education system. Ireland, on the other hand, was among the top countries due to the high number of graduates relative to the inputs used.
\end{Abstrakt}

\klicovaslova{
    data envelopment analysis, education, EU countries, efficiency
}


\clearpage