\hlavka{ A Panel Analysis of the Economic Determinants of Military Spending  }{ Michael Michael Kejmar }{ michael.kejmar@unob.cz }{ University of Defence, Czechia\newline University of Defence, Czechia\newline University of Defence, Czechia }{ Jiří Neubauer, Jakub Odehnal }

\begin{Abstrakt}
    The development of military spending in NATO countries is characterised by an increase in military spending caused mainly by the changing security situation in Europe. The security situation and the economic environment are considered as factors (determinants) influencing the size of military spending. The aim of the article is to present the possible use of the Dynamic Panel Data model (GMM) to identify military spending determinants of selected NATO countries. To analyse the determinants of military spending of 23 countries, the following economic variables describing the economic, fiscal development of a country were selected: the size of government expenditures, the size of the government budget surplus (deficit), the size of the country's indebtedness, the economic development measured by GDP, and the size of government revenues. The results of the Dynamic Panel Data model  confirm the positive effect of government surplus, government expenditures on military spending and negative effect of government debt on military spending. The results of the model confirm the expected hypotheses about the impact of selected economic variables on the military spending of 23 NATO countries in the period 1996-2002.
\end{Abstrakt}

\klicovaslova{
    GMM model, Economic Determinants, Military Spending
}


\clearpage