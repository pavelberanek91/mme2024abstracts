\hlavka{ Spatial Lag Model of the Real Estate Market in the Ústí nad Labem Region }{ Alena Pozdílková }{ alena.pozdilkova@uhk.cz }{ University of Pardubice, Czechia\newline University of Pardubice, Czechia }{ Jaroslav Marek }

\begin{Abstrakt}
    The aim of the article is spatial modeling of the relationship of apartment prices between neighboring municipalities in the Ústí nad Labem Region and a numerical study with an empirical demonstration of the model's applicability based on residual analysis. The spatial lag model will be used for the calculation and the unknown parameters will be estimated by the least squares method. This model is often used to describe geoinformation phenomena. Formulating an appropriate spatial regression model is not a simple task. The simultaneous determination of an unknown system of dependencies and estimating the spatial lag coefficient is challenging. The source data were obtained by automatically downloading data from real estate advertising websites. Every day from January 2019 to March 2024, data was collected on the floor area of the advertised apartments and the requested purchase price. Average prices per one square meter of an apartment were calculated from them. Our calculation used a matrix of spatial weights based on the nearest neighbor method. A graphical representation of the results explains the context in the spatial configuration.
\end{Abstrakt}

\klicovaslova{
    Spatial Lag Model, Spatial Modelling, Real Estate Market, Weights Matrix, Estimation of Unknown Parameters
}


\clearpage