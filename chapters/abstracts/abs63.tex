\hlavka{ The use of hypergraphs for recommender system design }{ Ladislav Beranek }{ beranek@ef.jcu.cz }{ University of South Bohemia, Faculty of Economics, Czechia\newline University of South Bohemia, Faculty of Economics, Czechia\newline University of South Bohemia, Faculty of Economics, Czechia\newline University of South Bohemia, Faculty of Economics, Czechia\newline University of South Bohemia, Faculty of Economics, Czechia }{ Radim Remes, Jiri Homan, Jan Fesl, Michal Konopa }

\begin{Abstrakt}
    Currently, recommendation systems are an integral part of e-commerce business platforms. E-commerce systems can obtain large amounts of customer data, such as which products users purchase or data that customers use to search for products. This data is used for personalized recommendations, to predict search trends, or to improve search results. In general, recommender systems deal with two problems. The first problem is a low number of purchases or searches related to specific prod-ucts, and the second is that queries may be directed more likely to only some popu-lar items. This second problem is referred to as disassortative mixing. To overcome these problems, we will use a hypergraph and a bipartite graph in our proposal of a prediction algorithm for a recommender system. It will allow us to utilize additional information from customer relationships. Our procedure treats all products appear-ing in the same customer session as a single hyperedge. We assume that a common purchasing interest unites items in customer sessions. It allows the initial bipartite graph to be transformed into a hypergraph. We conducted experiments with two e-commerce datasets. Experimental results show that the proposed solution provides good results compared to other algorithms.
\end{Abstrakt}

\klicovaslova{
    Network systems, E-commerce, Recommendation, Hypergraphs
}


\clearpage