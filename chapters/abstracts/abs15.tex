\hlavka{ Portfolio Analysis: Exploring Rank Length Metrics }{ Tereza Čapková }{ capkot00@jcu.cz }{ University of South Bohemia, Faculty of Economics, Czechia }{  }

\begin{Abstrakt}
    Portfolio analysis is a crucial aspect of financial management, with numerous specialists continually seeking to develop novel approaches that may enhance decision-making methods. This study investigates the application of Extreme Rank Length (ERL) and Continuous Rank Length (CRL) metrics as alternative approaches for assessing portfolios, deviating from traditional optimization methods. The motivation for this work stems from the robustness of stochastic dominance. To determine the effectiveness of ERL and CRL in evaluating portfolios, we simulate portfolios using the eleven most active stocks by dollar volume. The performance of these portfolios is evaluated using ERL and CRL metrics. Our research opens the door for further investigation and development in financial analysis by highlighting the potential of these metrics in portfolio evaluation.
\end{Abstrakt}

\klicovaslova{
    Portfolio Performance, GET package, ERL, CRL, Extreme Rank Length, Continuous Rank Length, Financial Management, Financial Analysis
}


\clearpage