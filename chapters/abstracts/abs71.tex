\hlavka{ Re-calculation of scientometric indicators of ORMS journals by non-traditional models.  }{ Josef Jablonský }{ jablon@vse.cz }{ Prague University of Economics and Business, Czechia }{  }

\begin{Abstrakt}
    Performance and citation impact of scientific journals are measured by traditional metrics such as impact factor, article influence score, journal citation indicator, and others. While the impact factor is based on the total number of citations and does not reflect the quality of journals cited, the article influence score considers the past importance of the citing journals. This paper aims the analysis and re-calculation of the performance of journals by data envelopment analysis (DEA) models. Traditional radial and SBM DEA models with weight restrictions where the outputs of the models are the citation counts from Q1 to Q4 categories and other journals according to the impact factor are applied. The results of the study are illustrated on the set of 80 journals from the Web of Science category Operational Research and Management Science (ORMS). The dataset for the study was obtained from the Journal Citation Reports in the period from 2017 until 2022. The relative efficiency scores and the ranking of the journals obtained by the models are presented and compared with traditional metrics.
\end{Abstrakt}

\klicovaslova{
    data envelopment analysis, journal impact factor, article influence score, ranking
}


\clearpage