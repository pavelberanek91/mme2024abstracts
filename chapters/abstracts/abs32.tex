\hlavka{ Almost Stochastic Dominance Analysis of Mean-Variance Efficient Portfolios }{ Jana Junová }{ junova@karlin.mff.cuni.cz }{ Charles University, Czechia\newline Charles University, Czechia }{ Miloš Kopa }

\begin{Abstrakt}
    Stochastic dominance is a tool that allows the comparison of random variables, representing the random returns of investments under very general assumptions. However, the generality of these assumptions can lead to situations where dominance between two random variables does not exist, even though the majority of investors evidently prefer one. For this reason, a relaxation of stochastic dominance called almost stochastic dominance was introduced. While the definition of almost first-order stochastic dominance is widely accepted, the definitions of almost second-order stochastic dominance (ASSD) vary.\newline This article aims to describe the different approaches to ASSD and analyze the relationships between them. Using data regarding the monthly returns of 49 industry representative portfolios, we find the mean-variance efficient portfolios and analyze their ASSD relationship to the minimum variance portfolio, employing and comparing different definitions of ASSD.
\end{Abstrakt}

\klicovaslova{
    stochastic dominance, almost stochastic dominance, portfolio optimization
}


\clearpage