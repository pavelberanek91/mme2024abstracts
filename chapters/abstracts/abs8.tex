\hlavka{ Application of the Three-Level Aggregation Model for Evaluating Opinions Under Hesitance for Fuzzy Voting in Spatial Planning Public Decision-Making }{ Benjamin Emmenegger }{ benjamin.emmenegger@unifr.ch }{ University of Fribourg, Human-IST Institute, Bd de Pérolles 90, CH-1700 Fribourg, Switzerland, Switzerland\newline VSB - Technical University of Ostrava, Ostrava, Czech Republic, Czechia\newline University of Fribourg, Human-IST Institute, Bd de Pérolles 90, CH-1700 Fribourg, Switzerland, Switzerland\newline University of Fribourg, Human-IST Institute, Bd de Pérolles 90, CH-1700 Fribourg, Switzerland, Romania\newline VSB - Technical University of Ostrava, Ostrava, Czech Republic, Czechia }{ Miroslav Hudec, Edy Portmann, Georgiana Bigea, Zapletal Frantisek }

\begin{Abstrakt}
    Diverse activities, such as construction, affect the living conditions of inhabitants and their subgroups differently. Citizens should express their support and/or resistance to each alternative. Since these opinions are highly subjective, it is reasonable to use fuzzy sets to express them and reveal to what extent an opinion inclines to these opposite poles. To tackle the challenge of receiving inconsistent responses, i.e., simultaneous high levels of support and resistance for the same alternative, this paper deals with reinforcing the consistent answers and, vice versa, weakening the contradictory responses.  We consider geographical subgroups depending on the degree to which each alternative would affect them. The impact of coalitions among subgroups is explored because it does matter if, e.g., two the most affected subgroups or two lightly affected subgroups agree. Using the selected fuzzy measures, we assign weights to subgroups and their coalitions based on geographical features. Additionally, to check the robustness of the results, a careful sensitivity analysis by Monte Carlo simulation is done. In this way, we emphasise the importance of understanding the dynamics within and between these subgroups for interpreting the results. The model accentuates differences in the data and offers a clearer view of the tendencies.
\end{Abstrakt}

\klicovaslova{
    fuzzy voting, choquet integral, spatial planning, decision making
}


\clearpage