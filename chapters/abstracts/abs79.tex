\hlavka{ The Impact of Language-Based AI on Wealth Inequality: Insights from Multi-Agent Simulations }{ Pavel Beránek }{ pavel.beranek@ujep.cz }{ Jan Evangelista Purkyně University in Ústí nad Labem, Czech University of Life Sciences Prague, Czechia }{  }

\begin{Abstrakt}
    This research explores the integration of Large Language Models (LLMs) in multi-agent simulations to study the effects of natural language bargaining on wealth distribution. Utilizing advanced language models—GPT-4 with 1.7 trillion parameters and the smaller Phi-2 with 3 billion parameters—this study seeks to uncover new insights into economic behaviors within the Boltzmann distribution model, enriching traditional economic models with sophisticated, realistic agent interactions. Through a series of simulations set in a grid-based network topology, we investigate how LLM-mediated negotiations impact wealth distribution among agents under varying agent attributes and external economic conditions. Our research questions specifically address the influence of LLM-based bargaining on wealth distribution through generated negotiation dialogues, the emergence and effectiveness of negotiation strategies, variations in the Gini coefficient across different simulation setups, and the capacity of LLMs to reveal emergent properties not detectable in traditional models. We anticipate our findings will encourage interdisciplinary collaboration among AI, economics, and social and psychological sciences. This study not only demonstrates the significant capabilities of language-based AI in modeling complex economic interactions but also emphasizes the pivotal role of communication in influencing economic outcomes, thereby offering essential insights for crafting more equitable economic systems.
\end{Abstrakt}

\klicovaslova{
    Boltzmann Wealth Distribution, Economic Inequality, AI in Economic Modeling, Multi-agent systems, Large-language models
}


\clearpage