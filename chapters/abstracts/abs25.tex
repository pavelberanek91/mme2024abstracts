\hlavka{ Designing optimal transportation patterns for radiation incident recovery }{ Robert Hlavatý }{ hlavaty85@gmail.com }{ CZU Prague, Czechia\newline CZU Prague, Czechia\newline CZU Prague, Czechia\newline CZU Prague, Czechia }{ Helena Brožová, Anna Selivanova, Tereza Sedlářová Nehézová }

\begin{Abstrakt}
    In case of a radiation accident in a nuclear power plant (NPP), there is a possibility of radionuclide releases that would affect the vicinity of the NPP. Such an event subsequently requires responsible authorities to decontaminate the affected areas and remove the produced contaminated waste to the designated interim storage sites. Given the extent of the incident, the recovery of the areas may turn into long-term logistic operation involving considerable machinery and personnel. We propose an optimal routing methodology to approach this situation and employ linear optimization to deliver an effective solution to the problem. This specific routing problem is constrained not only by the vehicle capacities but also by the doses the personnel can take during the process. We seek a solution that minimizes the distance travelled by vehicles and, thus, the time for which the personnel is exposed to radiation. The presented methodology is based on the real-world measures that would be taken in case of such an incident. The model results would allow the authorities to plan sufficient vehicle and personnel availability, estimate the time needed to clear contaminated areas and estimate the capacity of the interim storage sites. We demonstrate our methodology on a small example.
\end{Abstrakt}

\klicovaslova{
    dose rate, linear optimization, nuclear power plant, radiation incident, routing problem, waste
}


\clearpage