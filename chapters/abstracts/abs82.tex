\hlavka{ The Economic Effect of Advanced Delivery Locations for the Last Mile Routing Problem }{ Ece Yağmur }{ ecyagmur@ktun.edu.tr }{ Konya Technical University, Turkey }{  }

\begin{Abstrakt}
    The last-mile logistics phase is considered the most costly phase in the supply chain. So, some innovative delivery options have been proposed for last-mile activities in recent years. In real life, while some customers prefer only home delivery, others prefer to self-pick up their orders for a certain amount of discount from advanced delivery locations that are accessible 24/7. This paper analyzes a novel last-mile routing problem by utilizing lockers. For optimal distribution decisions under the objective of minimizing the weighted sum of total travel and tardiness costs, a new mixed integer programming (MIP) formulation is developed. The proposed model is applied to newly generated test instances, and some sensitivity analyses of selected parameters are investigated. According to the computational results, it is observed that the exact solver CPLEX is quite sensitive to the number of customers and lockers.
\end{Abstrakt}

\klicovaslova{
    vehicle routing, home delivery, advanced delivery locations, last mile logistics, mixed integer programming
}


\clearpage