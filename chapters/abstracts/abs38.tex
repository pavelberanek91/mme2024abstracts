\hlavka{ Assessing tourism efficiency in traditional beach touristic centers in Mexico: Application of dynamic two-stage DEA model and fuzzy time series forecasting }{ Martin Flegl }{ martin.flegl@tec.mx }{ School of Engineering and Sciences, Tecnologico de Monterrey, Mexico\newline Facultad de Turismo y Gastronomía, Universidad Anáhuac México, Mexico\newline School of Engineering and Sciences, Tecnologico de Monterrey, Mexico\newline Faculty of Business and Economics, Mendel University, Czechia }{ Carmen Lozano, Patrick Joaquín Cruz, Marketa Matulova }

\begin{Abstrakt}
    Tourism is one of the most important economic sectors in Mexico, because it has positioned itself as one of the main tourist destinations internationally and has promoted national, regional, and local development. In other words, the participation of the tourism sector went from contributing 6.9% in 2020 to 7.6% in 2021 of the Gross Domestic Product (GDP) at current prices. The arrival of national and international tourists to Mexico has been constantly growing during the last two decades. In 2022, Mexico registered arrival of 38.326 million of international tourists, ranking the country as the 6th most visited in the world. Constant growth of tourists arrivals resulted in a direct growth of a hospitality capacity across the country. At the end of 2022, Mexico offered 881,022 hotel rooms, representing 14.55% bigger capacity compared to 2016. Such a growth creates imminent pressure to guarantee efficiency in the whole tourism sector.\newline To assess the tourism efficiency in Mexico, we used the Data Envelopment Analysis, one of the most used methodologies for measuring efficiency and performance. We constructed a two-stage dynamic DEA model using monthly data for seven traditional beach touristic centers for a period 2015-2021. Stage 1 evaluates the hospitality efficiency considering hotels capacity and tourists arrival, whereas Stage 2 focuses on museums and archeological zones visits efficiency taking into account locations’ attractivity. Further, we combined the obtained efficiency results with fuzzy time series forecasting to expose a future efficiency trend in each traditional beach touristic center.
\end{Abstrakt}

\klicovaslova{
    Data Envelopment Analysis, Fuzzy logic, Mexico, Tourism, Window Analysis
}


\clearpage