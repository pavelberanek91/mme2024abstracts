\hlavka{ Accessibility of public charging infrastructure for electric vehicles across Central European countries: a geospatial analysis. }{ Tomas Formanek }{ formanek@vse.cz }{ VSE Prague, Czechia\newline VSE Prague, Czechia }{ Jindřich Lacko }

\begin{Abstrakt}
    This study models spatial distribution of electric vehicle (EV) charging stations across eight EU nations, a pivotal aspect in the decarbonization strategy of the transportation domain. Employing geolocated open source dataset, augmented with official statistical data, we conduct a quantitative analysis on a finely granulated hexagonal grid. Our findings demonstrate that zero-inflated two stage models exhibit superior performance compared to conventional Poisson count model in this specific context. Moreover, our modeling endeavors reveal that a substantial portion of the variability in the geographic dispersion of EV chargers can be explained by country-specific fixed effects and the population count of a given grid cell.
\end{Abstrakt}

\klicovaslova{
    Electric vehicles, charging infrastructure, spatial analysis, count data model
}


\clearpage