\hlavka{ Analyzing Determinants of Success in Ice Hockey World Championships }{ Vladimír Holý }{ vladimir.holy@vse.cz }{ Prague University of Economics and Business, Czechia }{  }

\begin{Abstrakt}
    The study examines the determinants of team success in the Ice Hockey World Championships. The annual final rankings are analyzed through a dynamic model utilizing the Plackett–Luce distribution with time-varying worth parameters driven by the conditional score, i.e. the gradient of the log-likelihood. Various exogenous variables are incorporated to address key questions: Does the host team enjoy a home advantage? Does the success of junior teams correlate with increased prospects of winning the main tournament in subsequent years? Are teams more advantaged by youthful talent or seasoned players? Furthermore, the study investigates which game statistics hold the most significance for accurate forecasting.
\end{Abstrakt}

\klicovaslova{
    Ice Hockey Rankings, Generalized Autoregressive Score Model, Plackett–Luce Distribution
}


\clearpage