\hlavka{ Critical Managerial Decision Making and Matrix Games }{ Jana Heckenbergerova }{ jana.heckenbergerova@upce.cz }{ University of Pardubice, Czechia }{  }

\begin{Abstrakt}
    A key part of a manager's job is to find the optimal solution. This can be difficult task without knowledge of advanced methods, data analysis, or evaluating the suitability of strategies. In these decision-making processes, game theory tools can be successfully used. In conflict situations, the optimal choice of strategy can lead to a win-win situation, i.e. to satisfaction on both sides. While game solutions in the field of pure strategies can be found using Nash equilibrium, matrix games in which a saddle point cannot be found lead to more complex linear programming tasks.\newline If we have enough information, for example, a competitive contest between two or more players in the market can be described using a specific matrix game. These will be the aim of this contribution. In a few selected practical examples, we will show how linear programming can be used in a managerial position. For each situation, we first demonstrate a mathematically formal description of the problem, then find the optimal solution and finally interpret the result from an economics perspective.\newline First described conflict is between two banking institutions regarding the level of interest rates. The second example will show whether a strike is an appropriate strategy for raising wages. The third conflict situation concerns workers under a new manager who wants to introduce weekend work. And the last presented managerial decision is the choice of a suitable future investment.
\end{Abstrakt}

\klicovaslova{
    conflict situation, decision making, optimization, matrix game, linear programming
}


\clearpage