\hlavka{ A Coalition Formation as a Multicriteria Voting Game }{ Michaela Tichá }{ Michaela.Ticha@ujep.cz }{ Jan Evangelista Purkyně University, Czechia\newline Prague University of Economics and Business, Czechia }{ Martin Dlouhý }

\begin{Abstrakt}
    In the multicriteria voting game, we assume a set of political parties and a set of po-litical programs with multiple dimensions of public policy. The coalition program is formulated as a weighted average of the individual political programs. The objec-tive of each political party is to minimize the maximum distance between the coali-tion program and its own program, to maximize its own share of power in the win-ning coalition, and to maximize the stability of the winning coalition, which is measured as the maximum distance between the coalition program and the indi-vidual programs of all political parties in the coalition. The multicriteria voting game model is applied to the Chamber of Deputies of the Parliament of the Czech Repub-lic. Deputies were surveyed using a questionnaire that included 16 key questions. For each question, deputies selected from five scaled responses. The data obtained was used to calculate the optimal winning coalition.
\end{Abstrakt}

\klicovaslova{
    game theory, multicriteria voting game, coalition stability, political power dynamics
}


\clearpage