\hlavka{ Application of conventional DEA and ZSG-DEA models in university budget allocation }{ Eva Štichhauerová }{ eva.stichhauerova@tul.cz }{ Technical University of Liberec, Czechia\newline Technická univerzita v Liberci, Czechia }{ Miroslav Žižka }

\begin{Abstrakt}
    This paper investigates the feasibility of applying two Data Envelopment Analysis (DEA) models in allocating budgets among departments in a university. The paper is divided into three parts. The first part describes the existing methodology of budget allocation in a selected faculty of a given university. Then, two DEA models that have been used in alternative budget construction are characterized. The first model is based on the classical CCR model, supplemented by the super-effectiveness model of Andersen and Petersen. The model redistributes resources from inefficient units towards efficient units, depending on the degree of their super-efficiency. The reallocation takes place until all units are efficient. The second model is based on the zero-sum gains (ZSG) approach, where the sum of the resources of all units remains constant. Thus, the sum of the improvement and the worsening of all units under study must remain equal to zero. The last part of the paper compares the current budget allocation with alternative options under the classical and ZSG-DEA models.
\end{Abstrakt}

\klicovaslova{
    Budget allocation, Data Envelopment Analysis, zero-sum gains DEA model, Andersen and Petersen super-efficiency model
}


\clearpage