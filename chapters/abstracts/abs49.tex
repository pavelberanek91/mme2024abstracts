\hlavka{ Model of network interconnections }{ Petr Fiala }{ pfiala@vse.cz }{ Prague University of Economics and Business, Czechia\newline University of Finance and Administration, Prague, Czechia }{ Renata Majovska }

\begin{Abstrakt}
    In today's economy, many activities are networked. The importance of network industries that deliver products and services is growing. Traditionally, only the effects of competition have been analysed within networks. Companies are increasingly finding that even the cooperation of competitors can bring benefits to all involved. Network interconnections are analysed, where networks not only provide services on their own network, but also allow access to foreign networks. The paper a model and analyses of co-opetition in network industries using biform games as a combination of the non-cooperative and cooperative games theory. The authors propose the division of the biform games into sequential and simultaneous games. Theproposed model can be solved as a sequential biform game.
\end{Abstrakt}

\klicovaslova{
    Network industry, Competition, Cooperation, Biform games
}


\clearpage