\hlavka{ Choking Hazard: Surviving the Heat of Competitive Counter-Strike }{ Jan Rejthar }{ rejj02@vse.cz }{ University of Economics and Business, Prague, Czechia }{  }

\begin{Abstrakt}
    This article delves into choking under pressure among Counter-Strike players, aiming to identify determinants of pressure in professional competitions. Employing regression analysis, the study examines whether players' performance is affected by the intense pressure often encountered in competitive gaming scenarios. Drawing from the complete competitive history of Counter-Strike: Global Offensive present at archive of hltv.org consisting of nearly 77,000 matches, the research reveals parallels with traditional sports psychology, showcasing that players indeed experience performance decline under pressure.\newline A key finding of the study is the positive mediating effect of experience on players' susceptibility to choking under pressure. Experienced players exhibit greater resilience to pressure-induced performance decrements compared to their less seasoned counterparts.\newline This research enhances our understanding of the psychological intricacies within esports, emphasizing the importance of experience in shaping player performance under pressure. By leveraging comprehensive data and rigorous regression analysis techniques, the study offers actionable insights for player development and performance optimization in competitive gaming contexts.
\end{Abstrakt}

\klicovaslova{
    Counter-Strike, Choking under pressure, Esports, Economics of sports
}


\clearpage