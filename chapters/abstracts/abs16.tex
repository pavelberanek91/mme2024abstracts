\hlavka{ Efficiency analysis of grass seed multiplication in Germany: A case study of the leading Company }{ Kevin Nowag }{ xnowag@node.mendelu.cz }{ PhD Student Mendel University, Germany }{  }

\begin{Abstrakt}
    In the past, multiplication was mainly focused on quality. Global political events have made market prices for competing agricultural products much more volatile and put prices in the multiplication sector under considerable pressure. This has brought the efficiency of multiplication much more into focus.\newline Data envelopment analysis is used for the efficiency analysis. The aim of the study is to compare the yields achieved with the help of basic potentials of the farmers with regard to soil fertility and weather data. In the next step of the analysis, the DEA results are used to perform a regression with specific parameters of the contract farmers. The analysis of the results should make it possible to characterize contract farmers and enable the seed company to increase efficiency by selecting farmers differently.\newline The data for the DEA are GPS coordinates of the investigated fields, weather data from the German weather service and the Müncheberger Soil Quality Rating was used for the potential of the soil. The yield achieved per field was scaled using the average multi-year yield of the specific variety in order to be able to compare the complete data.
\end{Abstrakt}

\klicovaslova{
    Efficiency analysis, Data envelopment analysis, Multiplication of grasses, Seed business, Agriculture
}


\clearpage