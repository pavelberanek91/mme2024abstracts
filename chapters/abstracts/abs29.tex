\hlavka{ Analysis of traffic accidents and weather in the Czech republic  }{ Blanka Bazsova }{ blanka.bazsova@vsb.cz }{ VSB – Technical University of Ostrava, Czechia\newline VSB – Technical University of Ostrava, Czechia }{ Lucie Chytilova }

\begin{Abstrakt}
    Many factors affect a traffic accident. It can be people's moods, overwork, inattention, alcohol, excessive speed, and the weather. Since the latter factor has been changing a lot recently in connection with climate change, it is necessary to determine whether these general assumptions exist and affect the accident rate positively or negatively. The theoretical assumption is that these climate changes are related to temperature fluctuations in the Czech Republic in summer and winter. We are increasingly experiencing more tropical nights and, conversely, very icy days and freezing spring, sweltering summer and virtually snowless winter. These factors associated with temperature fluctuations and the number of rainfall events are gaining importance. They are worth looking at in terms of their monthly evolution over the past seven years. This article examines the weather and accident rate in the Czech Republic. A model uses standard accident rate variables (death, serious and minor injuries or material damage) and weather-specific variables. More precisely, the investigated econometric model thus includes the mentioned standard variables and average temperature and precipitation. The relevant model is examined and tested using correlation and regression analysis and their assumptions. Moreover, based on a detailed analysis, the dependence between the accident rate and the weather is proven, and it can be seen that the change in weather generally affects the accident rate positively.
\end{Abstrakt}

\klicovaslova{
    corellation analysis, regression analysis, traffic accidents, weather
}


\clearpage