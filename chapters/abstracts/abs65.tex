\hlavka{ A Two-Stage DEA Model for Evaluating the Efficiency of SMEs with Multi-Year Accounting Data }{ Lucie Chytilová }{ chytilova@pef.czu.cz }{ Czech University of Life Sciences Prague, Czechia\newline VŠB - Technical university of Ostrava, Czechia }{ Hana Štverková }

\begin{Abstrakt}
    This study proposes the extension of the two-stage Data Envelope Analysis (DEA) model to assess the efficiency of small and medium enterprises (SMEs). The model leverages multi-year accounting data and incorporates a temporal dimension to capture the dynamic nature of SME operations. The primary focus is on evaluating these enterprises' stability and efficiency. The proposed model decomposes the evaluation process into two sub-stages: Stage 1 focuses on human capital efficiency, and Stage 2 assesses business efficiency. Outputs from Stage 1 serve as inputs for Stage 2, reflecting the sequential nature of these processes. These phases influence each other, even in different periods. Data from 2020 to 2022 are used. This approach allows for a more comprehensive evaluation by capturing the effectiveness of human capital utilisation (Stage 1) and its subsequent translation into business efficiency (Stage 2). The analysis will categorise SMEs into efficient and inefficient groups, further delving into efficiency levels at each stage. By examining the relationships between human capital, business skills, and overall efficicency, the study aims to identify key drivers of efficiency in SMEs. Finally, based on the findings, the research proposes practical recommendations for enhancing SME operations and developing effective business support mechanisms.
\end{Abstrakt}

\klicovaslova{
    Data Envelopment Analysis, two-stage, small and medium business, efficicency
}


\clearpage