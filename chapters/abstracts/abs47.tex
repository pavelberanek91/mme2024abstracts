\hlavka{ Improvement of Methods of Fertility Rates Modelling }{ Ondřej Šimpach }{ ondrej.simpach@vse.cz }{ University of Economics Prague, Czechia }{  }

\begin{Abstrakt}
    Fluctuations and trends of fertility rates development do not have to be regular or long-term. Knowledge of fertility rates is needed for policy planning and public ad-ministration.Therefore, we focus on the modelling of the fertility rates in the Czech Republic. Particularly, we apply standard Lee-Carter model with time-independent parame-ters ax and bx and time-varying index kt. Hyndman “demography” package in RStudio software is utilized. Because parameter ax (the average value of the empiri-cal time series) can be biased, we suggest an approach for its improvement.We utilize functional data (fertility rate, age, year), where fertility rate is a function of women's age for time period 1950 to 2022. There are 4 models with different ax parameter compared: a) standard parameter of the Lee-Carter model, b) median of age-specific fertility rates, c) ax calculated on data 1999–2022, and d) ax calculated on data 2008–2022. However, no approach was found to be better than the original calculated as simple arithmetic means of fertility rates in specific age which had the lowest mean squared error.These results are important for subsequent analyses because for working with de-mographic data about fertility it is important to consider the most recent data, which are not significantly skewed and influenced by a range of factors.
\end{Abstrakt}

\klicovaslova{
    fertility rates, Lee Carter model, policy planning, stochastic modelling
}


\clearpage