\hlavka{ Flexible Job Shop Scheduling with setup, transportation and planned machine idle time. }{ František Koblasa }{ frantisek.koblasa@tul.cz }{ Technical University of Liberec - Department of Manufacturing Systems and Automation, Czechia\newline Technical University of Liberec - Department of Manufacturing Systems and Automation, Czechia }{ Miroslav Vavroušek }

\begin{Abstrakt}
    The Flexible Job Shop Scheduling Problem (FJSP) is one of the most popular scheduling models because of its ability to describe various real-life manufacturing systems. Despite being used mainly in its natural form, more practical constraints such as transportation and setup times have attracted attention in the last decade as setups and internal transportation are the most visible non-valued added processes.\newline This article focuses on Flexible Job Shops with transport and setup times, adding planned idle times between machine operations. Those idle times depend not on job types as sequence-dependent setups but on machine type and represent regular maintenance, administration, scrap management, etc.\newline This article aims to enhance the known FJSP models with setup and transportation times by idle time constraints and test the real-world approach of dispatch-ing rules against the advanced evolution algorithm technique. The basic scheduling generation technique is compared with the earliest processing job start selection.\newline Together with the model, known FJSP testing instances are modified to suit the needs of the above-mentioned constraint and experiment. Generalization of testing instances modifications to real-world and combinatorial optimization needs is discussed.
\end{Abstrakt}

\klicovaslova{
    Flexible Job Shop Scheduling Problem, Transportation, Setup, Planned idle time, Evolution Algorithm
}


\clearpage