\hlavka{ A diffused and thickening soil monitoring model for evaluating the impact of climate change on agriculture }{ Imma Lory Aprea }{ immalory.aprea@uniparthenope.it }{ University of Naples Parthenope, Italy }{  }

\begin{Abstrakt}
    The sensor allocation problem and processing data detected by sensor devices represent a challenging topic in the agriculture field. More specifically, acquisition data in soil monitoring aims to provide information about the effects of climate change on soil health in terms of soil fertility, salinity, moisture levels, and nutrient levels. These soil structural parameters, which directly affect crop growth, give information about the monitoring system's reliability and allow the evaluation of the effectiveness of land management activities over time. This work proposes a diffused and thickening monitoring model to capture, according to an optimal approach, the most accurate information from sensor data, subject to a budget constraint and an environmental constraint. In detail, starting from a pre-existing network of zones, each controlled through a set of fixed diagnostic sensors, the aim is to identify further zones to be monitored by the same set of sensors to evaluate the soil health state. A geostatistical interpolation technique is used to estimate the soil structural parameters at any unsampled point of the area under analysis. These estimates are then used to measure the riskiness of extreme events, such as drought and floods, and then select the new zones to be added to the monitoring network. For the selection process, we consider a decision criterion based on the risk level of the occurrence of extreme phenomena under investigation and the comparison between the monitoring and non-monitoring costs for each zone of the examined area. Two types of constraints are involved: a budget constraint and an environmental constraint in order to limit the negative externalities that monitoring standard operations can have on the environment and provide the most sustainable monitoring network. This study describes the problem's resolution algorithm, and the zone selection criterion is tested using soil water temperature data in a dryland agricultural field.
\end{Abstrakt}

\klicovaslova{
    Risk-cost analysis, Sensor networks, Structural health monitoring, Sustainability criteria, Climate change
}


\clearpage