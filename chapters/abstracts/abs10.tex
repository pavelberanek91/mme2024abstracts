\hlavka{ Regional Intensity of the Freight: Functional Analysis of Variance }{ Peter Knížat }{ peter.knizat@euba.sk }{ University of Economics in Bratislava, Slovakia }{  }

\begin{Abstrakt}
    New digital data sources provide a great opportunity for econometrists to study the application of more complex statistical models, of which the corresponding empir-ical results can lead to prompt decision making since its early availability. In this paper, we introduce a theoretical framework of analysis of variance that is extend-ed to a functional space. An estimation procedure uses a modified method of least squares that minimizes a set of mathematical objects in its criterion. The estimated functional parameters are observed on a continuous domain rather than discrete point-wise estimates as its classical counterparts. In the empirical analysis, we use a dataset of electronic records that is collected from the satellite-based toll system in Slovakia. Each record refers to a passage of the vehicle through a section of the monitored road. We aggregate data into weekly time series, i.e. a number of pass-ing vehicles per week for each district. The weekly time series are transposed into a functional space through an expansion by basis splines. The observed mathematical objects that correspond to each district are categorized within a particular region and its co-variability is further analysed within the concept of functional analysis of variance. The main objective of the paper is to carry out the assessment of the intensity of the regional freight in Slovakia. The empirical results show various seasonal patterns of variations and significant differences of the freight between regions.
\end{Abstrakt}

\klicovaslova{
    regional freight, analysis of variance, functional parameters
}


\clearpage