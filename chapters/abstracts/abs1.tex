\hlavka{ Welfare or Poverty of Czech Pensioners during the Energy Crisis: A Linear Regression Model }{ Diana Bílková }{ bilkova@vse.cz }{ Prague University of Economics and Business, Czechia }{  }

\begin{Abstrakt}
    The paper is focused on the total net monthly income of the pensioner's household at the time of the energy crisis, which represents the explained quantitative variable in the linear regression model. This income represents one of the main characteristics of the quantitative aspect of living standards, and not only for households of senior citizens. The initial explanatory quantitative variables entering into the model are number of household members, property of the pensioner's family, number of living rooms, pensioner's age, municipality size, number of pensioner's children, age of the pensioner's partner and length of current partnership.The results of the sample survey are for the year 2022 and include only pensioners aged 65 and over. The data was provided by the Czech Statistical Office. The length of current partnership variable was removed from the model due to harmful multicollinearity. The sequential F-test showed that the most important explanatory variable is the number of household members in terms of influence on the explained variable.
\end{Abstrakt}

\klicovaslova{
    net monthly income of the pensioner's household, welfare and poverty of pensioners, negative effect of the energy crisis, multiple linear regression, stepwise regression, sequential F-test, multicollinearity, homoscedasticity and heteroscedasticity
}


\clearpage