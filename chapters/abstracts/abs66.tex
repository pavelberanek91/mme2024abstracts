\hlavka{ The Impact of Initial Price History on Asset Price Volatility: Insights from a Learning to Forecast Experiment }{ Michaela Sedláková }{ michaela.sedlakova@vsb.cz }{ VSB – Technical University of Ostrava, Faculty of Economics, Czechia }{  }

\begin{Abstrakt}
    The effect of the initial price history on asset price volatility is studied using a Learning to Forecast experiment (hereinafter LtF). Participants are tasked with predicting the future prices of three distinct risky assets over many consecutive periods. In contrast to previous LtF experiments that focused on a single risky asset, this research allows participants to compare the price trajectories of individual assets. One asset is characterized by a very stable initial price development compared to the other two assets. Given that all risky assets share the same fundamental value, we can investigate the effect of different initial price history on the overall price dynamics. Our conjecture is that the asset with a stable initial price history will exhibit lower volatility compared to the other two assets. This hypothesis is verified through statistical tests that are applied with respect to selected measures of volatility – relative absolute deviation from the fundamental price and variance. From the results, it is clear that the asset characterized by a stable initial price history in most cases demonstrates reduced price volatility. Moreover, we find that the initial price history significantly impacts participants’ coordination behaviour during the experiment.
\end{Abstrakt}

\klicovaslova{
    behavioral finance, experimental economics, expectations
}


\clearpage