\hlavka{ Asymmetries in Savings-Investment Nexus: Global Perspectives }{ Lukáš Frýd }{ lukas.fryd@vse.cz }{ Prague University of Economics and Business, Czechia }{  }

\begin{Abstrakt}
    This study investigates the intricate dynamics between savings and investment, focusing on potential asymmetries and heterogeneous effects across different investment quantiles in both Large and Non-Large economies. Building upon empirical evidence, we explore whether the relationship between savings and investment varies according to economic conditions, particularly during the growth or decline phases. In contrast to the symmetric effects commonly assumed, our analysis unveils significant asymmetries in the impact of savings shocks on investment changes, notably observing stronger connections for adverse shocks compared to positive ones. Moreover, we identify variations in this relationship across different investment quantiles and distinguish between the responses of Large and Non-Large economies. Specifically, adverse savings shocks demonstrate stronger associations with investment changes during periods of economic downturn or decapitalization, while positive shocks exhibit heightened effects during growth phases. These findings underscore the importance of recognizing asymmetries and quantile-specific effects, providing a nuanced understanding of the interplay between savings and investment dynamics.
\end{Abstrakt}

\klicovaslova{
    quantile regression, common latent factors, saving-investment nexus, asymmetries
}


\clearpage