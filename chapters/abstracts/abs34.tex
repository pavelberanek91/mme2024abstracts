\hlavka{ The Influence of the Global Recession of 2008 on Organic Food Consumption }{ Šárka Čížková }{ cizeko@vse.cz }{ University of Economics, Prague, Czechia }{  }

\begin{Abstrakt}
    The goal of the study is to analyze how the 2008 global economic crisis affected the consumption of organic food in the Czech Republic. The study is designed to explore overall consumption attitudes. Its objective is to measure the influence of certain macroeconomic indicators, particularly income-related ones, on general consumption of organic food. Additionally, it aims to analyze how this impact evolved before and after the 2008 crisis. From a methodological point of view, the error correction methodology (ECM) is implemented, albeit with modifications to integrate breakpoint analysis for modeling the effects of the global economic crisis on organic food consumption. The econometric analysis of the estimated model confirmed a statistically significant positive dependence between organic food consumption indicators and all examined income-related indicators before 2008. However, it is also showed that this dependence ceased to exist following the global economic recession in 2008. This retrospective examination offers valuable insights into the dynamics of organic food consumption shifts triggered by economic downturns.
\end{Abstrakt}

\klicovaslova{
    organic food, consumption function, income, economic crisis, error correction model
}


\clearpage