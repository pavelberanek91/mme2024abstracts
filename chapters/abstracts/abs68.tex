\hlavka{ The Energy Mix in Europe: A Panel Regression Analysis }{ Lucie Chytilová }{ lucie.chytilova@vsb.cz }{ Vysoká škola báňská – Technická univerzita Ostrava, Czechia\newline Vysoká škola báňská – Technická univerzita Ostrava, Czechia }{ Jana Hančlová }

\begin{Abstrakt}
    The energy mix in Europe is undergoing a significant transformation with an emphasis on decarbonisation and increasing energy security. This work examines trends in the EU's energy mix using panel regression. The analysis includes data from 28 EU countries from 2000-2022. It focuses on the development of energy consumption, energy production from renewable and non-renewable sources and dependence on energy imports.\newline The results show that energy consumption in the EU is growing slightly while energy production from renewable sources is rising significantly. Dependence on energy imports is decreasing but still high. Panel regression reveals that several factors influence the evolution of the energy mix in the EU, including economic growth, energy prices, climate policy, and energy security.\newline The thesis further examines specific trends in individual EU countries and identifies key challenges and opportunities for transforming the energy mix. In conclusion, it summarises the main results and proposes policy recommendations for achieving a sustainable and secure energy mix in Europe.
\end{Abstrakt}

\klicovaslova{
    Energy mix, Europe, Panel regression, Renewable Energy, Non-renewable energy sources
}


\clearpage