\hlavka{ The relationship between investment in machinery, employment, and output of forestry in Switzerland }{ Richard Kovárník }{ xkovarn1@mendelu.cz }{ Mendel University in Brno, Faculty of Business and Economics, Department of Statistics and Operation Analysis, Czechia }{  }

\begin{Abstrakt}
    This article deals with time series modelling in the field of forest management. Specifically, it is a time series containing information on investments in machinery and equipment, a time series on the number of work units and a time series on the output of the forestry sector. The data comes from the period between 1992 and 2022 in Switzerland. For this purpose, a vector autoregression model was estimated to model the dynamics between these variables, Granger causality was tested to determine interrelationships and causalities and Impulse-response analysis was performed to understand how quickly and significantly variables respond to shocks. The results obtained indicate that sudden changes in forestry output stimulate additional investment in the following year to meet the increased demand for logging and wood processing. Furthermore, the results indicate that investments in machinery and equipment affect the number of working units. In this direction, there is a decrease in working units if there were additional investments in machinery and equipment in the previous period.
\end{Abstrakt}

\klicovaslova{
    vector autoregression, Granger causality, impulse-response, forestry
}


\clearpage