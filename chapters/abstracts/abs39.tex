\hlavka{ Asset return as a vague element in investment portfolio selection: fuzzy mathematical modelling }{ Adam Borovička }{ adam.borovicka@vse.cz }{ Prague University of Economics and Business, Czechia }{  }

\begin{Abstrakt}
    The return, or its level, is often an unstable, or uncertain aspect of the intended investment. It can be expressed deterministically (e.g. by a mean) or stochastically (as a random variable with a particular probability distribution). The first option loses some valuable information. The second option can complicate subsequent, particularly computational, tasks when expressing a random process explicitly. Another possibility is to use the apparatus of fuzzy set theory. Return as a (triangular) fuzzy number – fuzzy return – can adequately quantify the uncertainty associated with its expected value. The triangular form offers several particular computational advantages. Their eventual comparison also proceeds more easily. Thus, processing such fuzzy information with a mathematical model for the selection of an investment portfolio need not be difficult. However, a crucial issue is the determination of the three parameters of the fuzzy number, which is sometimes neglected in papers on this topic, although it can logically significantly affect the result. The application power of the fuzzy return concept integrated into the mathematical programming model is demonstrated through a case study of ESG mutual fund portfolio selection.
\end{Abstrakt}

\klicovaslova{
    fuzzy return, mutual fund, portfolio selection, triangular fuzzy number
}


\clearpage