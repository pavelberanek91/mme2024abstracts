\hlavka{ Evaluation of the efficiency of national energy markets in 27 EU countries in 2011-2022 }{ Jana Hanclova }{ jana.hanclova@vsb.cz }{ VSB-Technical University of Ostrava, Czechia\newline VSB-Technica University of Ostrava, Czechia\newline VSB-Technical University of Ostrava, Czechia }{ Lucie Chytilova, Dominika Bordacsova }

\begin{Abstrakt}
    The European energy market is characterized by a diverse mix of energy sources, including fossil fuels, nuclear power, and an increasing share of renewable energy sources such as wind, solar, and biomass. The market operates within the framework of the European Union (EU), with regulations aimed at promoting competition, ensuring security of supply, and reducing greenhouse gas emissions.\newline The contribution is devoted to the evaluation of the efficiency of national energy markets in 27 EU countries in the years 2011-2022. Attention is paid to the share of renewable sources and undesirable production of emissions. The evaluation is based on a modified model of data envelopment analysis with undesirable outputs. We use direct distance function, non-proportional changes, and increasing returns to scale. The results document that the critical output is precisely the share of renewable resources and that there is no significant improvement in the examined period. The second critical output is the production of emissions, where the situation usually improves over the years. The analysis contributes to revealing the causes of the development of unfavorable trends in the energy markets of EU countries.
\end{Abstrakt}

\klicovaslova{
    data envelopment analysis, energy market, European Union, renewable energy, greenhouse gas emissions
}


\clearpage