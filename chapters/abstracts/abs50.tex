\hlavka{ Identification, Generation, and Evaluation of Intralogistic Solutions }{ Dominika Bordácsová }{ bor0151@vsb.cz }{ Vysoká škola báňská - Technická univerzita Ostrava, Czechia\newline Vysoká škola báňská - Technická univerzita Ostrava, Czechia\newline Vysoká škola báňská - Technická univerzita Ostrava, Czechia }{ Jana Hančlová, Lucie Chytilová }

\begin{Abstrakt}
    This contribution focuses on the identification and generation of intralogistic solution combinations that meet specific customer requirements. The classical framework and processes are innovated based on expert experiences and the future development of the company in a competitive environment. The proposed system utilizes historical data and expert knowledge while respecting customer requirements when establishing individual scenarios. For each scenario, an intralogistic solution that minimizes the final cost while adhering to all constraints is found through optimization. For the most probable scenario, a distribution including risk will be sought, and changes in the solution's performance will be monitored. Performance values will be compared with remaining scenarios for modeling the time intensity of picking within the designed intralogistic solutions. The obtained results enable the identification of the impacts of time intensity on performance within the examined scenarios. In this way, we allow for a more realistic and precise simulation of intralogistic processes. A critical part of our method is the evaluation of various scenarios, which enables us to quantify the impacts of uncertainty in picking speed on the performance of automated systems and to identify the optimal solution for each unique customer situation. This work offers a new perspective on planning and managing intralogistics by combining stochastic modeling with practical market needs. Our analysis results demonstrate how efficient and adaptive intralogistic systems can be when properly designed and calibrated with consideration for the inherent uncertainties of operations. Our findings provide a valuable foundation for further research and development in the field of automated intralogistics and offer practical guidelines for implementation in real logistic operations.
\end{Abstrakt}

\klicovaslova{
    Intralogistics, Optimization, Scenario Analysis
}


\clearpage