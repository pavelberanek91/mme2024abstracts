\hlavka{ A neural network-particle swarm solver for sustainable portfolio optimization problems }{ Imma Lory Aprea }{ immalory.aprea@uniparthenope.it }{ University of Naples "Parthenope", Italy\newline University of Trieste, Italy }{ Gabriele Sbaiz }

\begin{Abstrakt}
    In this talk, we present sustainable portfolio optimization problems in which we aim to maximize a refinement to the Sharpe Ratio measure and to minimize a systemic risk measure. More specifically, we first consider the Modified Sharpe Ratio: it corresponds to the standard Sharpe Ratio when the excess rate of return is positive; while it is adjusted by multiplying the previous quantity by the standard deviation, if such an excess rate of return is negative. Furthermore, the considered systemic risk measure is represented by the Delta Conditional Value at Risk, a tail-dependence measure meant to quantify the potential losses of a portfolio due to the riskiness associated with an individual asset. In addition, in the optimization problem, we take into account two types of real-world trading constraints. On the one hand, we impose stock market restrictions through buy-in thresholds and budget constraints. On the other hand, a turnover threshold restricts the total amount of trades allowed in the rebalancing phases. Finally, in order to meet the growing appetite for sustainable investments, we impose a green threshold into the portfolio’s construction. To deal with these asset allocation models, we embed a suitable hybrid constraint-handling procedure into an improved Particle Swarm Optimizer (PSO) that is dynamically adjusted by a neural network architecture. It is worth noting that implementing the neural network paradigm is fundamental for enhancing the PSO’s performance and improving the quality of estimating the Modified Sharpe Ratio and Delta Conditional Value at Risk measures. Finally, we conduct empirical tests on different data sets to illustrate the effectiveness of the proposed strategies and evaluate the performance of our investments as the sustainable preferences vary.
\end{Abstrakt}

\klicovaslova{
    Particle Swarm Optimizer, Neural Networks, Sustainable portfolios, Constrained Optimization Problems
}


\clearpage