\hlavka{ Enhancing Sustainability in Waste Collection - A Clustering Approach }{ Katarzyna Gdowska }{ kgdowska@agh.edu.pl }{ AGH University of Krakow, Poland\newline AGH University of Krakow, Poland }{ Martyna Zaucha }

\begin{Abstrakt}
    This study aims to enhance waste collection practices' sustainability in Tarnów, which currently relies on a diverse fleet of vehicles. Addressing the need for efficient strategic planning, the research focuses on devising a solution that ensures equitable workload distribution among service providers. Central to this endeavor is the clustering of municipal waste collection points into sectors, fostering a balanced utilization of the fleet across the city.\newline Drawing on principles of sustainable development, the optimization problem revolves around determining how to cluster collection points to promote equitable resource utilization and minimize environmental impact. By achieving uniform fleet consumption and mitigating greenhouse gas emissions, the proposed approach contributes to the broader goal of reducing environmental harm associated with internal combustion vehicles.\newline The Mixed Integer Linear Programming (MILP) was used as the research methodology. The study emphasizes the creation of clusters aligned with specified criteria, facilitating optimal fleet management. Computational experiments utilizing real-world data from the Tarnów Municipality validate the efficacy of the proposed model.
\end{Abstrakt}

\klicovaslova{
    solid waste manageemnt, fleet management, clustering, mixed-integer programming
}


\clearpage