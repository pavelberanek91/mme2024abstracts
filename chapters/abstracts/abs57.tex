\hlavka{ Sensitivity of genders to economic fluctuations }{ Martin Boďa }{ martin.boda@umb.sk }{ Matej Bel University in Banská Bystrica, Slovakia\newline Matej Bel University in Banská Bystrica, Slovakia\newline Jan Evangelista Purkyně in Ústí nad Labem, Czechia\newline Jan Evangelista Purkyně in Ústí nad Labem, Czechia }{ Mariana Považanová, Michal Struk, Michaela Tichá }

\begin{Abstrakt}
    For OECD countries, the paper studies whether it is male or female labour force that is more sensitive to fluctuations in overall economic activity. Towards this end, a two-stage procedure is applied. First, Okun's law is estimated in its unemployment-based and employment-based version in a time-varying framework for both males and females. Second, the estimated Okun coefficients are matched against sectoral and labour market characteristics of OECD economies, their demographic make-up and other explanatory factors. The Okun coefficients net of structural factors confirm that males are indeed more sensitive to business cycles. In comparison to the extant research, a more refined econometric procedure is employed and more robust findings are established.
\end{Abstrakt}

\klicovaslova{
    Okun law, gender differences, sensitivity to economic fluctuations
}


\clearpage