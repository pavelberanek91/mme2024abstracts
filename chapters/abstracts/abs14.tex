\hlavka{ Profit Allocation in a Multi-Echelon Closed-Loop Supply Chain: A Cooperative Game Theoretical Approach }{ Petr Pokorný }{ petrpokorny@yahoo.ca }{ Prague University of Economics and Business, Czechia }{  }

\begin{Abstrakt}
    In this paper we consider a CLSC consisting of a manufacturer that uses both new and recycled materials to manufacture products sold by a retailer. A third collector collects the end-of-life products and sells them to the manufacturer for reprocessing. In contrast to the dominant non-cooperative research, we use the cooperative game theory approach to study the stability of the coalition structures. We start with a non-cooperative Stackelberg solution led by the producer and then form all possible coalitions. By analyzing the core and its stability conditions, we prove that the core is not empty and that the end customer can benefit from better net product prices when coalitions are formed. The fairness of the profit distribution is tested using the Shapley value and is shown to be in the core. The return rate all increases under the cooperative approach, with one exception when the third party collector is not part of any coalition.
\end{Abstrakt}

\klicovaslova{
    Closed-Loop Supply Chain, Cooperative, Game Theory, Nash Equilibrium, Core, Shapley
}


\clearpage