\hlavka{ Extension of Planning Poker by Work Contour Models in Project Management }{ Jan Bartoška }{ bartoska@pef.czu.cz }{ Czech University of Life Sciences Prague, Department of Systems Engineering, Czechia\newline Czech University of Life Sciences Prague, Department of Systems Engineering, Czechia\newline Czech University of Life Sciences Prague, Department of Systems Engineering, Czechia }{ Josef Kunhart, Jiří Pilný }

\begin{Abstrakt}
    The paper describes an extension of Scrum Planning Poker with Work Contour Model . Planning Poker is used by the agile team to determine the difficulty of tasks without determining the work effort in tasks. Work effort variability affects the team: As it increases, team member cooperation and proactivity decreases and agile principles may be compromised in the team. The authors of the paper propose adding another characteristic to the planning poker based on work effort models. The proposal builds on previous research by one of the authors on quantifying Student's syndrome and work contours of work effort, with using these modifications for Earned Value Management. Labeling tasks on Kanban Board with new characteristics added to the Planning Poker may affect cooperation on tasks during Scrum. Tasks marked as "last-minute work" should have higher priority for team collaboration. The paper includes a case study for use in practice and proposes a new concept of the mathematical model for work contours and work effort in agile teams. The proposed concept extends the Planning Poker and enhances adherence to the agile principles in the team.
\end{Abstrakt}

\klicovaslova{
    Project management, Agile approach, Planning Poker, Work Contour, Work Effort, Resource Allocation, Scrum
}


\clearpage